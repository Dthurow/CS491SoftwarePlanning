\documentclass[11pt]{article}
\usepackage{enumerate}
\usepackage{amsfonts}
\usepackage{graphicx}
\usepackage{tabularx}

\pagestyle{empty} \setlength{\parindent}{0mm}
\addtolength{\topmargin}{-0.5in} \setlength{\textheight}{9in}
\addtolength{\textwidth}{1.75in} \addtolength{\oddsidemargin}{-0.9in}


\title{Sumerian Project\\
Software Requirements Specification\\
CS 491}
\author{Danielle Thurow \\
Seoung Jung\\
Zachary Fox\\
Thomas Fritchman}
\date{}


\begin{document}
\maketitle
\newpage

\tableofcontents
\newpage

\begin{center}
\Large Revision History\\
\begin{tabularx}{\textwidth}{|c|X|c|}
    \hline
    \textbf{Revision} & \textbf{Description} & \textbf{Date}\\ \hline
    1.0 & Initial Document & 6/6/2014\\ \hline
\end{tabularx}
\end{center}
\newpage

\section{INTRODUCTION}
\subsection{PURPOSE}
For assyriologists, consistently identifying months and years when the tablets were created is a vital part of recreating a Sumerian social network. The Date Extrapolator will solve this problem in two parts, and will be integrated with the work done by the previous group. 

The first part is the implementation of an algorithm. This will identify the years and months, and will take into account things such as damage and scribes’ transcription errors. The conclusions of the algorithm will be tested against training data to analyze correctness. It will also accept outside input to attempt to correct any conclusions the algorithm has made. There will be a relatively easy way to adjust the parameters of the algorithm.

The second part of the solution is user interface. This will allow the user to give input on the algorithm’s correctness. It will also display statistics about the algorithm’s correctness.

\subsection{INTENDED AUDIENCE AND READING SUGGESTIONS}
Readers of this document include any persons involved in the development of this program as a whole. This document will outline all requirements necessary, pertaining to the Date Extrapolator program, and will create a bridge between the customer and designers, in order to create an agreement on what is crucial to the overall success of this project. 

\subsection{PRODUCT SCOPE}
\subsubsection{SCOPE OF INITIAL RELEASE}
Release 1 will be:
FE-01 not implemented
FE-02 not implemented
FE-03 not implemented
FE-04 implement way to tweak algorithm for our specific algorithm
FE-05 basic web page that displays statistics information
FE-06 A simple implementation of a semi-naive algorithm

\subsubsection{SCOPE OF SUBSEQUENT RELEASES}
Release 2 will be:
FE-01 not implemented
FE-02 provide webpage that displays algorithmically-chosen dates from text for confirmation,
FE-03 adds confirmed dates to database and tweaks algorithm
FE-04 no update
FE-05 give the statistics a better GUI for webpage
FE-06 improve implementation/ statistics for our algorithm

Release 3 will be:

FE-01 implement completely
FE-02 tweak and improve
FE-03 tweak and improve
FE-04 tweak and improve
FE-05  tweak and improve
FE-06  tweak and improve

\subsection{REFERENCES}
www.wwucuneiform.com
https://sites.google.com/site/wwucsseniorprojectcuneiform/

\section{OVERALL DESCRIPTION}
\subsection{PRODUCT PERSPECTIVE}
It is an extension of an existing project aiming to improve the functionality of the current software.

\subsection{CONTEXT DIAGRAM}
%TODO

\subsection{USER CLASSES AND CHARACTERISTICS}
\begin{itemize}
    \item Expert User: an expert user is allowed to confirm or disagree with the algorithm’s suggested date, as well as do everything a normal user can.
    \item Normal user: A normal user is allowed to see the algorithm’s suggested dates and statistics of the current algorithm.
\end{itemize}

\subsection{OPERATING ENVIRONMENT}
This program is to be built into the currently existing web application.

\subsection{ASSUMPTIONS AND DEPENDENCIES}
AS-1: experts input is noncontroversial and correct
AS-2: there is enough time
AS-3: pattern exists in sumerian date to create efficient algorithm

DE-1: depending on work of previous group’s work


\section{EXTERNAL INTERFACE REQUIREMENTS}
\subsection{USER INTERFACES}
Navigation controls will have a similar feel to the rest of the existing product.
UI details will be included in another specification

\subsection{SOFTWARE INTERFACES}
PHP for website
Java for database
Python for cleaning the data





\end{document}
