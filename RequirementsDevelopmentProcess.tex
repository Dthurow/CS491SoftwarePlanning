\documentclass[11pt]{article}
\usepackage{enumerate}
\usepackage{amsfonts}
\usepackage{graphicx}
\usepackage{tabularx}

\pagestyle{empty} \setlength{\parindent}{0mm}
\addtolength{\topmargin}{-0.5in} \setlength{\textheight}{9in}
\addtolength{\textwidth}{1.75in} \addtolength{\oddsidemargin}{-0.9in}


\title{Sumerian Project\\
Requirements Development Process\\
CS 491}
\author{Danielle Thurow \\
Seoung Jung\\
Zachary Fox\\
Thomas Fritchman}
\date{}


\begin{document}
\maketitle
\newpage

\tableofcontents
\newpage

\begin{center}
\Large Revision History\\
\begin{tabularx}{\textwidth}{|c|X|c|}
    \hline
    \textbf{Revision} & \textbf{Description} & \textbf{Date}\\ \hline
    1.0 & Initial Document & 5/9/2014\\ \hline
\end{tabularx}
\end{center}
\newpage

\section{PURPOSE}
The purpose of the Requirements Development document is to solidify a concrete process that ensures both the customer and the developers embark on a valuable and feasible project. A formal process is helpful to all parties and decreases chances of miscommunication greatly.

\section{STAKEHOLDERS}
\begin{itemize}
    \item Dr. Garfinkle of WWU History Department
    \item Aran Clauson
    \item Cuneiform group 1.
\end{itemize}

\section{PROCESS}

\subsection{ELICITATION}
\textbf{Purpose:} Obtain a direction from the customer.\\
\\
\textbf{Entry Criteria: }Have a customer with a need or want.\\
\\
\textbf{Tasks:} \\
Write a Vision and Scope Document \\
Define requirement development process\\
Define vision and scope\\
Identify user classes. \\
Identify use cases. \\
Identify system events and responses. \\
Observe users performing their jobs\\
\\
\textbf{Exit Criteria:} All requirements have been assessed and are documented.\\
\\
\textbf{Work Products:} \\
Vision and Scope Document.\\
Use cases document.\\

\subsection{ANALYSIS}
\textbf{Purpose:} Assess feasibility of customer request. \\
\textbf{Entry Criteria:} We know what the customer wants. \\
\\
\textbf{Tasks:} \\
Draw context diagram \\
Create prototype \\
Analyze feasibility \\
Prioritize requirements \\
Model the requirements \\
Apply quality function deployment \\
Assess reuasability and potential for expansion \\
\\
\textbf{Exit Criteria:} All requests have been approved and finalized by group. \\
\\
\textbf{Work Products:} \\
Context Diagram \\
Prioritized Requirements \\
Risk and Opportunity Study \\

\subsection{SPECIFICATION}
\textbf{Purpose:} To formally document the customer requirements.\\
\\
\textbf{Entry Criteria:} The statement of work has been received and all requests have been approved by group. \\
\\
\textbf{Tasks:} \\
Adopt SRS template \\
Identify sources of requirements \\
Uniquely label each requirement \\
Record business rules \\
Specify quality attributes \\
\\
\textbf{Exit Criteria:} All requests have been approved and finalized by group. \\
\\
\textbf{Work Products:} \\
SRS \\
Supplementary Specification \\

\subsection{VALIDATION}
\textbf{Purpose:} To ensure our specification meets our customer's needs and is a feasible project. \\
\\
\textbf{Entry Criteria:} Completed Specification \\
\\
\textbf{Tasks:} \\
Validate SRS \\
Define Acceptance Criteria \\
Test requirements \\
\\
\textbf{Exit Criteria: } All documents approved by team and customer. \\
\\
\textbf{Work Products:} \\
Requirements Validation Plan \\
Prototype \\

\section{VERIFICATION}
Team will review documents and ensure correctness. The documents will be submitted to the customer for approval.

\section{EXIT CRITERIA}
Documents approved by customer.

\section{REFERENCES}
\textit{Software Requirements 2nd Edition}, Karl E Weigers (p. 45) Microsoft Press, Redmond, Washington, 2003
\\
Algae Counting Team Requirements Development Process V1.0

\end{document}
