\documentclass[11pt]{article}
\usepackage{enumerate}
\usepackage{amsfonts}
\usepackage{tabularx}

\pagestyle{empty} \setlength{\parindent}{0mm}
\addtolength{\topmargin}{-0.5in} \setlength{\textheight}{9in}
\addtolength{\textwidth}{1.75in} \addtolength{\oddsidemargin}{-0.9in}

\setcounter{tocdepth}{3}
\renewcommand{\contentsname}{Table of Contents}

\title{Sumerian Project\\
		Requirements Management Process\\
		CS 491}
\author{Danielle Thurow \\
		  Seoung Jung\\
		  Zachary Fox\\
		  Thomas Fritchman}
\date{}


\begin{document}
\maketitle

\newpage

\tableofcontents

\newpage

\begin{center}

\Large Revision History\\
\begin{tabularx}{\textwidth}{|c|X|c|}
	\hline
	\textbf{Revision} & \textbf{Description} & \textbf{Date}\\ \hline
	1.0 & Initial Document & 5/6/2014\\ \hline
\end{tabularx}


\end{center}

\newpage



\section{PURPOSE} 
The purpose of Requirements Management is to create an agreement between the customer and the project team over the customer's requirements the project team will address.\\
 
The purpose of this document is to define a Requirements Management Process is to track the requirements, control any changes, and to create baseline requirements.

\section{STAKEHOLDERS}  

\begin{itemize}

\item \textbf{Project Members -} Project members are people who are members the assigned team doing the current project. 

\item \textbf{Customer -} The customer is the person or group of people who have decided to initiate this particular project.

\item \textbf{End User -} The end user are anyone who will use the end result of this project.

\end{itemize}


\section{ACTIVITES} 

	
\subsection{Change Control}
\subsection{Version Control} 
\subsection{Requirements Status Tracking}
\subsection{Requirements Tracing} 
   
\section{VERIFICATION}
The team will each look over and approve each document before submission.

\section{EXIT CRITERIA} 
All documents have been approved by the entire team, and each have been put into our version control system.

\section{REFERENCES} 

	Algea Counting Team,
	\emph{Requirements Management Process}. \\
	CS491, Bellingham, Washington, 2013\\

  Karl E Weigers,
  \emph{Software Requirements. 2nd edition}. \\
  Microsoft Press, Redmond, Washington, 2003



\end{document}