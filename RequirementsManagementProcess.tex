\documentclass[11pt]{article}
\usepackage{enumerate}
\usepackage{amsfonts}
\usepackage{tabularx}

\pagestyle{empty} \setlength{\parindent}{0mm}
\addtolength{\topmargin}{-0.5in} \setlength{\textheight}{9in}
\addtolength{\textwidth}{1.75in} \addtolength{\oddsidemargin}{-0.9in}

\setcounter{tocdepth}{3}
\renewcommand{\contentsname}{Table of Contents}

\title{Sumerian Project\\
		Requirements Management Process\\
		CS 491}
\author{Danielle Thurow \\
		  Seoung Jung\\
		  Zachary Fox\\
		  Thomas Fritchman}
\date{}


\begin{document}
\maketitle

\newpage

\tableofcontents

\newpage

\begin{center}

\Large Revision History\\
\begin{tabularx}{\textwidth}{|c|X|c|}
	\hline
	\textbf{Revision} & \textbf{Description} & \textbf{Date}\\ \hline
	1.0 & Initial Document & 5/6/2014\\ \hline
	1.1 & Initial Document with activities filled out & 5/11/2014\\ \hline
\end{tabularx}


\end{center}

\newpage



\section{PURPOSE} 
The purpose of Requirements Management is to create an agreement between the customer and the project team over the customer's requirements the project team will address.\\
 
The purpose of this document is to define a Requirements Management Process is to track the requirements, control any changes, and to create baseline requirements.\\

\section{STAKEHOLDERS}  

\begin{itemize}

\item \textbf{Project Members -} Project members are people who are members the assigned team doing the current project. \\

\item \textbf{Customer -} The customer is the person or group of people who have decided to initiate this particular project.\\

\item \textbf{End User -} The end user are anyone who will use the end result of this project.\\

\end{itemize}


\section{ACTIVITES} 

\subsection{Change Control}

\begin{description}

\item[Purpose] \hfill \\
Define a process for adding changes.

\item[Entry Criteria]\hfill \\
The Requirements Development Process is under version control.

\item[Tasks]\hfill \\
Define a process to change requirements. \\
Create a change control board

\item[Exit Criteria]\hfill \\
Documents have been reviewed and placed under version control.  

\item[Work Products]\hfill \\
Change control process\\

\end{description}

\newpage
\subsection{Version Control} 

\begin{description}

\item[Purpose] \hfill \\
Create baselines of documents, and create a specific set up to keep track of revisions and new requirement version changes.


\item[Entry Criteria]\hfill \\
Documents for Requirement Development Process are under version control, any changed requirements have been approved by the Change Control Board


\item[Tasks]\hfill \\
Create a file naming convention.\\
Create a file repository.\\
Create baseline documents.\\
Control versions of requirements documents.\\


\item[Exit Criteria]\hfill \\
Documents have been reviewed by the team, revised, and approved. All of these documents have been put into the version control.
  

\item[Work Products]\hfill \\
Version naming convention.
\\


\end{description}

\subsection{Requirements Status Tracking}

\begin{description}

\item[Purpose] \hfill \\
To have a process for tracking the status of requirements, to keep track of which are most important, and to prevent errors from constantly changing status of requirements.


\item[Entry Criteria]\hfill \\
Documents for Requirement Development Process are under version control, any changed requirements have been approved by the Change Control Board

\item[Tasks]\hfill \\
Track the status of each requirement.\\
Keep a history of requirements status and changes\\


\item[Exit Criteria]\hfill \\
Documents have been reviewed by the team, revised, and approved. All of these documents have been put into the version control

\item[Work Products]\hfill \\
Requirements Scope Change Log


\end{description}

\newpage

\subsection{Requirements Tracing} 

\begin{description}

\item[Purpose] \hfill \\
To keep track of requirements, to make sure there are no overlapping requirements, and keeping track of which requirements are linked to each other.

\item[Entry Criteria]\hfill \\
Documents for Requirement Development Process are under version control, any changed requirements have been approved by the Change Control Board

\item[Tasks]\hfill \\
Populate a requirement traceability matrix.

\item[Exit Criteria]\hfill \\
The traceability matrix has been reviewed by the team and 
updated. It has been placed in version control.

\item[Work Products]\hfill \\
Requirements traceability matrix updated.

\end{description}
   
\section{VERIFICATION}
The team will each look over and approve each document before submission.\\

\section{EXIT CRITERIA} 
All documents have been approved by the entire team, and each have been put into our version control system.\\

\section{REFERENCES} 

	Algea Counting Team,
	\emph{Requirements Management Process}. \\
	CS491, Bellingham, Washington, 2013\\

  Karl E Weigers,
  \emph{Software Requirements. 2nd edition}. \\
  Microsoft Press, Redmond, Washington, 2003



\end{document}